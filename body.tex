\section{Introduction}
Parsec\cite{parsec} is a parser combinator library for Python heavily inspired by the Haskell library with the same name. It is capable of creating parsers for context-sensitive grammars that require infinite lookahead. The library provide some trivial parsers (e.g. parsing a single letter, digit or a string etc.), but the real power of the library is the parser combinators it provides. A parser combinator is a higher order function, which take a number of parsers as input and produce a new parser as output (i.e. it creates a new parser by combining existing parsers). Parsers created by combinator functions can further be used to create new parsers. This technique makes it easy to construct powerful, modular and well-structured parsers that are easy to test. Parsec make good use of Pythons support for operator overloading to make it easy to construct parsers in a concise and readable way.


% TODO: Write something about ease of testing
% TODO: It is inspired by Haskell's parsec library
% TODO: use parenthesis or not?
% TODO: Parsec provide some trivial parsers and powerful combinators.


\section{External Testing}
\section{External testing of Parsec.py}
% A draft document providing some external tests. This should be done before  you have a meeting in week 49.


\begin{python}
def test_letter(self):
    parser = letter()
    for c in "ABCDEFGHIJKLMNOPQRSTUVWXYZabcdefghijklmnopqrstuvwxyz":
        self.assertEqual(parser.parse(c), c)
    for c in "0123456789+-.,;:?^*":
        self.assertRaises(ParseError, parser.parse, c)

    self.assertEqual(parser.parse("xyz1"), "x")
    self.assertRaises(ParseError, parser.parse, "42")
\end{python}

\begin{python}
def test_times(self):
    parser = times(string("x"), 3, 5)
    self.assertRaises(ParseError, parser.parse, "")
    self.assertRaises(ParseError, parser.parse, "x")
    self.assertRaises(ParseError, parser.parse, "xx")
    self.assertEqual(parser.parse("xxx"), ["x"]*3)
    self.assertEqual(parser.parse("xxxx"), ["x"]*4)
    self.assertEqual(parser.parse("xxxxx"), ["x"]*5)
    self.assertEqual(parser.parse("xxxxxx"), ["x"]*5) # one x remains to be consumed
    self.assertRaises(ParseError, parser.parse_strict, "xxxxxx")
    self.assertRaises(ParseError, parser.parse_strict, "xxxxxxx")
    self.assertRaises(ParseError, parser.parse_strict, "xxxxxxxxxxxxxxxx")
\end{python}



\begin{python}
def test_joint(self):
    parser = string('x') + string('y')
    self.assertEqual(parser.parse('xy'), ('x', 'y'))
    self.assertRaises(ParseError, parser.parse, 'y')
    self.assertRaises(ParseError, parser.parse, 'z')

    nonlocals = {'changed': False}

    @generate
    def fn():
        nonlocals['changed'] = True
        yield string('y')

    parser = string('x') + fn
    self.assertRaises(ParseError, parser.parse, '1')
    self.assertEqual(nonlocals['changed'], False)
\end{python}



% \section{External listing highlighting}

% \pythonexternal{sample.py}

% \section{Inline highlighting}

% Definition \pythoninline{class MyClass} means \dots


\section{Proposed Changes}
While being able to create and combine parsers in Python using Parsec has its benefits it is sometimes easier to be able to express what to parse using a grammar. So our proposal is to extend Parsec to be able to read a grammar file where the user can specify a context free grammar and generate a parser from that. Our extended library would read the grammar file and generate a Python file as output containing the code for the parser.\par
Our motivation for chosing this as our project is that is well suited for test driven development. We will parse the grammar file using Parsec. And a Parsec parser is easy to test as it is very modular and we can perform whitebox testing on the individual parts that the parser is constructed with. The parser that gets generated from the grammar can be tested using black box testing.\par
We have decided to put some limitations on the grammars that we accept. These being that the grammar has to be left factored and does not contain any left recursion (direct or indirect).\par
In the grammar file it is possible to create tokens to give name to strings, that can be used as terminals in the grammar. A rule in the grammar is defined by giving the rule a name, followed by an equals sign and a list of productions separated by pipes ($\vert$). Which rule is the start rule has to be explicitly stated by setting the $start$ variable to be equal to the intended start rule. It is also possible to create comments that will be ignored by the parser by starting a line with a hash character (\#). More over, every non-comment line has to be ended with a semicolon (;). Below is an example grammar file.
\begin{python}
# Tokens
token number = \d+;
token plus = "+";

start = E;

# Grammar
E = TX;
T = "(" E ")" | num Y;
X = plus E | "";
Y = "*" T  | "";
\end{python}


\section{Implementation}
The implementation of the grammar-parser consist of four parts, the abstract syntax tree, the grammar parser, the semantic analyzer and code generator.  
The abstract syntax tree, which is an in-memory representation of the structure of a grammar.  
 The parser analyze a grammar and construct a in-memory representation, an abstract syntax tree of the grammar.  The parser is construct using parsec, which allowed it to be very modular and well-suited for test driven development.  
The semantic analyzer inspect the abstract syntax tree for any semantic error such as multiple definitions for an identifier or references to non-existing constructs. The code generator traverse an abstract syntax tree, analyze and emit code able to parse anything that conformes to the grammar represented by the abstract syntax tree.


\subsection{Abstract syntax tree}

The abstract syntax tree consist mainly of two types of nodes, values and declarations. Value nodes hold values. There are three types of values: strings, identifiers, and regular expressions. String literals are stored by \textit{String} nodes. references to other tokens and rules are stored by \textit{Identifier} nodes. Literals for regular expressions are stored in \textit{RegularExpression} nodes.\par
There are three different types of declaration nodes: \textit{Token}, \textit{Rule} and \textit{Start}. Token nodes hold token declarations, i.e. assign a name to a string literal or regular expression. Rule nodes store data about a rule declaration, i.e. the name of the rule and its productions. Start nodes holds information about which rule to start parsing from.

\pythonexternal{ast.py}

The root of the abstract syntax tree is the \textit{Grammar} node. It stores all the declarations of a grammar as well as comments. Comments are stored by \textit{Comment} nodes.




This is the parent for all nodes in the abstract syntax tree. It redefines the equality operators $==$ and $!=$ in order to allow nodes to be compared stucturally.
\pythonexternal{node.py}


\subsection{Parsing the grammar}
For parsing the grammar we have created a few different parsers, these parsers include token, comment, production, identifier and start. These parsers were created using a combination of Parsec's buildin parsers. For example, to make the parser find the start of the grammar, it first looks for the string "start", followed by an equals sign and then it will try to parse an identifier, finally followed by a semicolon.
\begin{python}
start = (symbol("start") >> symbol("=") >> identifier
    << symbol(";")).parsecmap(Start)
\end{python}
The other parsers are created in similar fashion, where they are just combinations of really small simple parsers. Because the parsers we have written are so modular it has been very easy to do this using test driven development. Since we have been able to split the job of parsing an entire grammar into many smaller parts, to acually parse a grammar is just a matter of calling the many parser with the smaller ones separated by try\_choice (try\_choice is a parser that takes several parsers and try to use them in the order they are given).
\begin{python}
grammar = many(trim(comment ^ start ^ token ^ rule)).parsecmap(Grammar)
\end{python}



\subsection{Parser generation}
The generation of the Python code is rather simple as most of the work of identifying the structure and semantics of the grammar is done with the parser. As the parser generates iterable objects from the Abstract Syntax Tree classes we can easily identify what the parser has found.


\subsubsection{Parsing Details}

The main idea is that we first parse the grammar and store the result of each line in the grammar file in data structures. Simultaneously constructing the core structure of the intended function body (parsec parser) that corresponds to every declaration in the grammar.

There is limited error handling during the construction of the Python code. The idea to this is that it is easier to either handle the errors with the parser or let the Python interpreter explain the faults that is generated. What we do is make sure there are no multiple definitions of any rule, token or the starting parser.

Comments in the grammar are thrown away. There is not really an easy way to distinguish if and which comments are wanted in the generated Python file or kept in the grammar. Due to this ambiguity we decided to throw away all comments.

\subsubsection{Code Construction}

The constraints (left-recursive, left-factored e.t.c.) on the grammar is the primary reason the translation to Python code is straightforward. All parsec generators are constructed with the decorator Generate from the Parsec library. This decorator allows us to recursively construct and name our parsers however we want.

This recursive descent is where we previously found a bug in how the try\_choice method works.



\subsection{Shortcomings}
While we have accomplished alot of what set out to do with this project, we still have a few features that we did not manage to complete in time.\par
One of the things we included in our project proposal was to be able to specify a regular expression in the grammar file, our implementation of this feature is incomplete due to the complexity involved in parsing a regular expression that is valid in Python. There are many corner cases that needs to be handled and we felt that the time we would have to spend on that could be spent on more important parts as this is not really a core feature anyway.\par
There are also some features that we did not state in our proposal, but that we would have liked to add if we had more time. The first one being able to specify actions to take place when a production is parsed.\par
We would also like to relax the contraints on the grammar. Right now we have put a limitation on the grammar forcing it be left factored and not have any left factoring. We would have liked to implement an algorithm that accepts a grammar without these limitations and transform it into something that is easy to work with.

% \section{Specification of changes}

% what?
\subsection{Create a parsec from grammar}

% why?
\subsection{Best of both worlds?}

% how?
\subsection{How it works?}

\subsubsection{Grammar}
%% Grammar to parser


% \section{Task specification}

% The important thing to remember is that all documents that you hand in are drafts towards the final report. The final report should contain

% \begin{itemize}
%     \item A description of your chosen library.
%     \item External tests demonstrating the behaviour of the library (black box testing).
%     \item Proposed changes or additions and their motivation
%     \item Implementation of proposed changes and test cases. The test cases should be both black box on the behaviour of the implementation and white box providing some code coverage.
% \end{itemize}

% In your final document all tests should come with some motivation of why you
% have the test and what you are trying to test.

% Deadlines and deliverables.

% \begin{itemize}
%     \item A meeting with your lab assistant (me or Anke) in week 49 (next week)
%     \item A draft document providing some external tests. This should be done before  you have a meeting in week 49.
%     \item A presentation Dec. 9th. Where you will present your proposed changes.
%     \item A draft document with your proposed changes.
%     \item Dec. 21/2 is the deadline for the final report. By this time all your changes must be implemented, all test cases must be documented.
% \end{itemize}
% It is important to know that you can hand in a draft report at any
% time to get feedback.


% example of how to cite stuff in the references.bib files
%Random citation \cite{the_lexer_hack} embeddeed in text.

\newpage

\bibliography{references}
\bibliographystyle{ieeetr}
