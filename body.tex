\section{Introduction}
Parsec\cite{parsec} is a parser combinator library for Python heavily inspired by the Haskell library with the same name. It is capable of creating parsers for context-sensitive grammars that require infinite lookahead. The library provide some trivial parsers (e.g. parsing a single letter, digit or a string etc.), but the real power of the library is the parser combinators it provides. A parser combinator is a higher order function, which take a number of parsers as input and produce a new parser as output (i.e. it creates a new parser by combining existing parsers). Parsers created by combinator functions can further be used to create new parsers. This technique makes it easy to construct powerful, modular and well-structured parsers that are easy to test. Parsec make good use of Pythons support for operator overloading to make it easy to construct parsers in a concise and readable way.


% TODO: Write something about ease of testing
% TODO: It is inspired by Haskell's parsec library
% TODO: use parenthesis or not?
% TODO: Parsec provide some trivial parsers and powerful combinators.


\section{External Testing}
\section{External testing of Parsec.py}
% A draft document providing some external tests. This should be done before  you have a meeting in week 49.


\begin{python}
def test_letter(self):
    parser = letter()
    for c in "ABCDEFGHIJKLMNOPQRSTUVWXYZabcdefghijklmnopqrstuvwxyz":
        self.assertEqual(parser.parse(c), c)
    for c in "0123456789+-.,;:?^*":
        self.assertRaises(ParseError, parser.parse, c)

    self.assertEqual(parser.parse("xyz1"), "x")
    self.assertRaises(ParseError, parser.parse, "42")
\end{python}

\begin{python}
def test_times(self):
    parser = times(string("x"), 3, 5)
    self.assertRaises(ParseError, parser.parse, "")
    self.assertRaises(ParseError, parser.parse, "x")
    self.assertRaises(ParseError, parser.parse, "xx")
    self.assertEqual(parser.parse("xxx"), ["x"]*3)
    self.assertEqual(parser.parse("xxxx"), ["x"]*4)
    self.assertEqual(parser.parse("xxxxx"), ["x"]*5)
    self.assertEqual(parser.parse("xxxxxx"), ["x"]*5) # one x remains to be consumed
    self.assertRaises(ParseError, parser.parse_strict, "xxxxxx")
    self.assertRaises(ParseError, parser.parse_strict, "xxxxxxx")
    self.assertRaises(ParseError, parser.parse_strict, "xxxxxxxxxxxxxxxx")
\end{python}



\begin{python}
def test_joint(self):
    parser = string('x') + string('y')
    self.assertEqual(parser.parse('xy'), ('x', 'y'))
    self.assertRaises(ParseError, parser.parse, 'y')
    self.assertRaises(ParseError, parser.parse, 'z')

    nonlocals = {'changed': False}

    @generate
    def fn():
        nonlocals['changed'] = True
        yield string('y')

    parser = string('x') + fn
    self.assertRaises(ParseError, parser.parse, '1')
    self.assertEqual(nonlocals['changed'], False)
\end{python}



% \section{External listing highlighting}

% \pythonexternal{sample.py}

% \section{Inline highlighting}

% Definition \pythoninline{class MyClass} means \dots


\section{Proposed Changes}
While being able to create and combine parsers in Python using Parsec has its benefits it is sometimes easier to be able to express what to parse using a grammar. So our proposal is to extend Parsec to be able to read a grammar file where the user can specify a context free grammar and generate a parser from that. Our extended library would read the grammar file and generate a Python file as output containing the code for the parser.\par
Our motivation for chosing this as our project is that is well suited for test driven development. We will parse the grammar file using Parsec. And a Parsec parser is easy to test as it is very modular and we can perform whitebox testing on the individual parts that the parser is constructed with. The parser that gets generated from the grammar can be tested using black box testing.\par
We have decided to put some limitations on the grammars that we accept. These being that the grammar has to be left factored and does not contain any left recursion (direct or indirect).\par
In the grammar file it is possible to create tokens to give name to strings, that can be used as terminals in the grammar. A rule in the grammar is defined by giving the rule a name, followed by an equals sign and a list of productions separated by pipes ($\vert$). Which rule is the start rule has to be explicitly stated by setting the $start$ variable to be equal to the intended start rule. It is also possible to create comments that will be ignored by the parser by starting a line with a hash character (\#). More over, every non-comment line has to be ended with a semicolon (;). Below is an example grammar file.
\begin{python}
# Tokens
token number = \d+;
token plus = "+";

start = E;

# Grammar
E = TX;
T = "(" E ")" | num Y;
X = plus E | "";
Y = "*" T  | "";
\end{python}


\section{Implementation}

% \section{Specification of changes}

% what?
\subsection{Create a parsec from grammar}

% why?
\subsection{Best of both worlds?}

% how?
\subsection{How it works?}

\subsubsection{Grammar}
%% Grammar to parser


% \section{Task specification}

% The important thing to remember is that all documents that you hand in are drafts towards the final report. The final report should contain

% \begin{itemize}
%     \item A description of your chosen library.
%     \item External tests demonstrating the behaviour of the library (black box testing).
%     \item Proposed changes or additions and their motivation
%     \item Implementation of proposed changes and test cases. The test cases should be both black box on the behaviour of the implementation and white box providing some code coverage.
% \end{itemize}

% In your final document all tests should come with some motivation of why you
% have the test and what you are trying to test.

% Deadlines and deliverables.

% \begin{itemize}
%     \item A meeting with your lab assistant (me or Anke) in week 49 (next week)
%     \item A draft document providing some external tests. This should be done before  you have a meeting in week 49.
%     \item A presentation Dec. 9th. Where you will present your proposed changes.
%     \item A draft document with your proposed changes.
%     \item Dec. 21/2 is the deadline for the final report. By this time all your changes must be implemented, all test cases must be documented.
% \end{itemize}
% It is important to know that you can hand in a draft report at any
% time to get feedback.


% example of how to cite stuff in the references.bib files
Random citation \cite{the_lexer_hack} embeddeed in text.

\newpage

\bibliography{references}
\bibliographystyle{ieeetr}
